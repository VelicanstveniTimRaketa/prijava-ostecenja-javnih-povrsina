\chapter{Specifikacija programske potpore}
		
	\section{Funkcionalni zahtjevi}
			
			\noindent \textbf{Dionici:}
			
			\begin{packed_enum}
				
				\item Regitrirani korisnik
					\begin{packed_enum}
						\item Klijent
						\item Administrator
					\end{packed_enum}
				\item Neregistrirani (anonimni) korisnik
				\item Razvojni tim
				\item Gradski ured
				
			\end{packed_enum}
			
			\noindent \textbf{Aktori i njihovi funkcionalni zahtjevi:}
			
			
			\begin{packed_enum}
				\item  \underbar{Klijent (inicijator) može:}
				
				\begin{packed_enum}
					
					\item Prijaviti se u sustav koristeći svoj email i lozinku
					\item Odjaviti se iz sustava
					\item Pregledati osobne podatke
					\item Mijenjati osobne podatke
					\item Izbrisati svoj profil
					\item Podnijeti prijavu preko web sučelja koja sadrži naziv, opis, geografske koordinate, te opcionalno fotografiju
					\item Odabrati koordinate preko karte ili unijeti najbližu adresu
					\item Povezati svoju prijavu na postojeću (ako takva postoji)
					\item Vidjeti promjene u statusu svoje prijave
					\item Odabrati aktivne prijave na karti
					\item Imati uvid u sve prijave u sustavu
					\item Pregledati povijet svojih prijava
					\item Filtrirati pregled prijava po lokaciji i temi
					\item Mijenjati podatke aktivne prijave u povijesti prijava
				\end{packed_enum}
				
				\pagebreak
			
				\item  \underbar{Administrator (inicijator) može:}
				
				\begin{packed_enum}
					
					\item Imati uvid u popis svih prijava
					\item Imati uvid u popis svih registriranih profila
					\item Uređivati podatke prijava
					\item Brisati prijave
					\item Imati uvid u popis svih gradskih ureda
					\item Uklanjati (brisati) korisničke profile
				\end{packed_enum}
				
				\item  \underbar{Neregistrirani korisnik (incijator) može:}
				
				\begin{packed_enum}
					\item Napraviti novi korisnički račun klijenta
					\item Podnijeti prijavu preko web sučelja koja sadrži naziv, opis, geografske koordinate, te opcionalno fotografiju
					\item Odabrati koordinate preko karte ili unijeti najbližu adresu
					\item Povezati svoju prijavu na postojeću (ako takva postoji)
					\item Imati uvid u sve prijave na stranici, te iste grupirati po temi i lokaciji

				\end{packed_enum}
				
				\item  \underbar{Baza podataka (sudionik) može:}
				
				\begin{packed_enum}
				\item Imati popis svih postojećih korisničkih računa
				\item Sadržavati sve jedinstvene oznake prijava neregistriranih oznaka
				\item Pohraniti svaku prijavu
					
				\end{packed_enum}
			\end{packed_enum}
			\eject 
			
			
				
			\subsection{Obrasci uporabe}
				
				\textbf{\textit{dio 1. revizije}}
				
				\subsubsection{Opis obrazaca uporabe}
					

					\noindent \underbar{\textbf{UC01 - Registracija korisnika u sustav}}
					\begin{packed_item}
	
						\item \textbf{Glavni sudionik: }Neregistrirani korisnik
						\item  \textbf{Cilj:} Stvoriti korisnički račun
						\item  \textbf{Sudionici:} Baza podataka
						\item  \textbf{Preduvjet:} Null
						\item  \textbf{Opis osnovnog tijeka:}
						
						\item[] \begin{packed_enum}
	
							\item Klijent bira opciju "registracija" na sučelju web aplikacicje
							\item Klijent unosi tažene podatke
							\item Korisnik je upisan u bazu podataka
						\end{packed_enum}
						
						\item  \textbf{Opis mogućih odstupanja:}
						
						\item[] \begin{packed_item}
	
							\item[2.a] Klijent unosi neispravni/postojeći username ili email
							\item[] \begin{packed_enum}
								
								\item Sustav obavještava korisnika o problemu i briše mu unesena polja
								
								\item Korisnik mijenja podatke u ispravne i registracija uspješno se privede kraju
								
							\end{packed_enum}
							
						\end{packed_item}
					\end{packed_item}
					
					\noindent \underbar{\textbf{UC02 - Unos nove prijave u sustav}}
					\begin{packed_item}
	
						\item \textbf{Glavni sudionik: }Korisnik
						\item  \textbf{Cilj:} Podnijeti novu prijavu
						\item  \textbf{Sudionici:} Baza podataka
						\item  \textbf{Preduvjet:} Null
						\item  \textbf{Opis osnovnog tijeka:}
						
						\item[] \begin{packed_enum}
	
							\item Korisnik upisuje tražene podatke pri unosu prijave 
							\item Sustav javlja ako postoji vremenski bliska prijava na toj lokaciji
							\item Korisnik može povezati svoju prijavu na postojeću (ako takva postoji)
							\item Prijava se predaje i zapisuje u sustav
						\end{packed_enum}
						
						\item  \textbf{Opis mogućih odstupanja:}
						
						\item[] \begin{packed_item}
	
							\item[1.a] Korisnik nije naveo sve zahtijevane podatke
							\item[] \begin{packed_enum}
								
								\item Sustav obavještava korisnika o problemu i javlja mu da popuni tražena polja
								
							\end{packed_enum}
							
						\end{packed_item}
					\end{packed_item}
					
					\pagebreak
					
					\noindent \underbar{\textbf{UC03 - Pregled prijava}}
					\begin{packed_item}
	
						\item \textbf{Glavni sudionik: }Korisnik
						\item  \textbf{Cilj:} Pregled postojećih prijava
						\item  \textbf{Sudionici:} Baza podataka
						\item  \textbf{Preduvjet:} Null
						\item  \textbf{Opis osnovnog tijeka:}
						
						\item[] \begin{packed_enum}
	
							\item Korisnik otvara pregled svih postojećih prijava
							\item Korisniku se nudi opcija filtriranja po temi i lokaciji
							\item Na sučelju se prikazuju filtritrane prijave
						\end{packed_enum}
					\end{packed_item}
					
					\noindent \underbar{\textbf{UC04 - Prijava u sustav}}
					\begin{packed_item}
	
						\item \textbf{Glavni sudionik: }Registrirani korisnik
						\item  \textbf{Cilj:} Prijaviti se svojim profilom u sustav
						\item  \textbf{Sudionici:} Baza podataka
						\item  \textbf{Preduvjet:} Registracija
						\item  \textbf{Opis osnovnog tijeka:}
						
						\item[] \begin{packed_enum}
	
							\item Korisnik upisuje korisničko ime i lozinku 
							\item Sustav javlja potvrdu ispravnosti unesinih podataka
							\item Korisniku se učitava njemu prilagođeno sučelje
						\end{packed_enum}
						
						\item  \textbf{Opis mogućih odstupanja:}
						
						\item[] \begin{packed_item}
	
							\item[1.a] Korisnik krivo unio korisničko ime/lozinku
							\item[] \begin{packed_enum}
								
								\item Sustav obavještava korisnika o problemu i javlja mu da ispravi tražena polja
								
								
							\end{packed_enum}
							
						\end{packed_item}
					\end{packed_item}
					
					\pagebreak
					
					\noindent \underbar{\textbf{UC05 - Uređivanje podataka prijave}}
					\begin{packed_item}
	
						\item \textbf{Glavni sudionik: }Administrator
						\item  \textbf{Cilj:} Korigirati podatke vezane za odabranu prijavu
						\item  \textbf{Sudionici:} Baza podataka
						\item  \textbf{Preduvjet:} Dodijeljena prava administratora
						\item  \textbf{Opis osnovnog tijeka:}
						
						\item[] \begin{packed_enum}
	
							\item Prikazuju se sve prijave
							\item Administrator može filtrirati prijave
							\item Administrator izmjenjuje podatke prijave
							\item Izmjenjena prijava se sprema u bazu podataka
						\end{packed_enum}
					\end{packed_item}
					
					\noindent \underbar{\textbf{UC06 - Brisanje prijave}}
					\begin{packed_item}
	
						\item \textbf{Glavni sudionik: }Administrator
						\item  \textbf{Cilj:} Obrisati određenu prijavu
						\item  \textbf{Sudionici:} Baza podataka
						\item  \textbf{Preduvjet:} Dodijeljena prava administratora
						\item  \textbf{Opis osnovnog tijeka:}
						
						\item[] \begin{packed_enum}
	
							\item Administratoru se prikazuje pregled prijava  
							\item Administrator bira prijavu koju želi izbrisati
							\item Prijava se uklanja iz baze podataka i više nije viidljiva u aplikaciji
							
						\end{packed_enum}
					\end{packed_item}
					
					\noindent \underbar{\textbf{UC07 - Pregled korisnika}}
					\begin{packed_item}
	
						\item \textbf{Glavni sudionik: }Administrator
						\item  \textbf{Cilj:} Uvid u profile registriranih korisnika
						\item  \textbf{Sudionici:} Baza podataka
						\item  \textbf{Preduvjet:} Dodijeljena prava administratora
						\item  \textbf{Opis osnovnog tijeka:}
						
						\item[] \begin{packed_enum}
	
							\item Korisnik bira opciju za pregled svih profila
							\item Otvara mu se lista svih registriranih profila
							
						\end{packed_enum}
					\end{packed_item}
					
					\pagebreak
					
					\noindent \underbar{\textbf{UC08 - Pregled osobnih podataka}}
					\begin{packed_item}
	
						\item \textbf{Glavni sudionik: }Klijent
						\item  \textbf{Cilj:} Pregledati osobne podatke svog profila
						\item  \textbf{Sudionici:} Baza podataka
						\item  \textbf{Preduvjet:} Registracija
						\item  \textbf{Opis osnovnog tijeka:}
						
						\item[] \begin{packed_enum}
	
							\item Korisnik ulazi u opis svog profila
							\item Sustav mu prikaže username, e-mail i lozinku
							
						\end{packed_enum}
					\end{packed_item}
					
					\noindent \underbar{\textbf{UC09 - Odjava}}
					\begin{packed_item}
	
						\item \textbf{Glavni sudionik: }Registrirani korisnik
						\item  \textbf{Cilj:} Odjaviti se iz sustava
						\item  \textbf{Sudionici:} Baza podataka
						\item  \textbf{Preduvjet:} Aktivna prijava
						\item  \textbf{Opis osnovnog tijeka:}
						
						\item[] \begin{packed_enum}
	
							\item Korisnik bira opciju odjava 
							\item Sustav ga vraća na početnu stranicu web aplikacije
						\end{packed_enum}
					\end{packed_item}
					
					\noindent \underbar{\textbf{UC10 - Brisanje kroisničkog računa}}
					\begin{packed_item}
	
						\item \textbf{Glavni sudionik: }Administrator
						\item  \textbf{Cilj:} Obrisati određeni račun
						\item  \textbf{Sudionici:} Baza podataka
						\item  \textbf{Preduvjet:} Dodijeljena prava administratora
						\item  \textbf{Opis osnovnog tijeka:}
						
						\item[] \begin{packed_enum}
	
							\item Administrator bira profil koji želi ukloniti
							\item Sustav ga za provjeru pita da potvrdi odluku
							\item Administrator potvrđuje i profil se uklanja iz baze podataka
						\end{packed_enum}
					\end{packed_item}
					
					\pagebreak
					
					\noindent \underbar{\textbf{UC11 - Uvid povijest prijava}}
					\begin{packed_item}
	
						\item \textbf{Glavni sudionik: }Klijent
						\item  \textbf{Cilj:} Pregledati svu povijest privedenih prijava
						\item  \textbf{Sudionici:} Baza podataka
						\item  \textbf{Preduvjet:} Registracija
						\item  \textbf{Opis osnovnog tijeka:}
						
						\item[] \begin{packed_enum}
	
							\item Korisnik bira opciju za prikazivanje povijesti prijava 
							\item Sustav mu na sučelju prikazuje sve njegove prijave
							\item Korisnik dodatno može filtrirati iste po temi i lokaciji
						\end{packed_enum}
						
					\end{packed_item}
					
					\noindent \underbar{\textbf{UC12 - Pregled gradskih ureda}}
					\begin{packed_item}
	
						\item \textbf{Glavni sudionik: }Administrator
						\item  \textbf{Cilj:} Pregledati popis svih postojećih gradskih ureda
						\item  \textbf{Sudionici:} Baza podataka
						\item  \textbf{Preduvjet:} Dodijeljena prava administratora
						\item  \textbf{Opis osnovnog tijeka:}
						
						\item[] \begin{packed_enum}
	
							\item Administrator odabire opciju za pregled svih gradskih ureda zapisanih u sustavu
							\item Sustav mu na sučelje prikazuje gradske urede zapisane u bazi podataka
						\end{packed_enum}
					\end{packed_item}
					
					\noindent \underbar{\textbf{UC13 - Povezivanje na postojeću prijavu}}
					\begin{packed_item}
	
						\item \textbf{Glavni sudionik: }Korisnik
						\item  \textbf{Cilj:} Vezati se ne vremenski blisku prijavu
						\item  \textbf{Sudionici:} Baza podataka
						\item  \textbf{Opis osnovnog tijeka:}
						
						\item[] \begin{packed_enum}
	
							\item Korisnik predaje prijavu
							\item Sustav mu javlja za vremenski blisku prijavu na toj lokaciji
							\item Korisnik bira hoće li se vezati na postojeću prijavu ili kreirati vlastitu
						\end{packed_enum}
					\end{packed_item}
					
					\pagebreak
					
					\noindent \underbar{\textbf{UC14 - Odabir aktivnih prijava sa karte}}
					\begin{packed_item}
	
						\item \textbf{Glavni sudionik: }Klijent
						\item  \textbf{Cilj:} Pogladati na karti neriješene prijave
						\item  \textbf{Sudionici:} Baza podataka
						\item  \textbf{Preduvjet:} Postojanje aktivnih prijava
						\item  \textbf{Opis osnovnog tijeka:}
						
						\item[] \begin{packed_enum}
	
							\item Klijent na karti odabire opciju za uvid u aktivne prijave 
							\item Sustav mu sve lokacije aktivnih prijava prikazuje na karti
						\end{packed_enum}
						
						\item  \textbf{Opis mogućih odstupanja:}
						
						\item[] \begin{packed_item}
	
							\item[1.a] U sustavu nema aktivnih prijava
							\item[] \begin{packed_enum}
								
								\item Sustav korisnika izbacuje iz pregleda karte i javlja mu da nema aktivnih prijava	
							\end{packed_enum}
							
						\end{packed_item}
					\end{packed_item}
					
					\noindent \underbar{\textbf{UC15 - Mijenjanje sadržaja aktivne prijave}}
					\begin{packed_item}
	
						\item \textbf{Glavni sudionik: }Klijent
						\item  \textbf{Cilj:} Promijeniti sadržaj vlastite aktivne prijave
						\item  \textbf{Sudionici:} Baza podataka
						\item  \textbf{Preduvjet:} Postojanje aktivne prijave
						\item  \textbf{Opis osnovnog tijeka:}
						
						\item[] \begin{packed_enum}
	
							\item Klijent odabire svoju aktivnu prijavu koju želi izmjeniti
							\item Klijent mijenja atribut u prijavi
							\item Prijava s ažuriranim podakom se zapisuje u bazu podataka
						\end{packed_enum}
						
						\item  \textbf{Opis mogućih odstupanja:}
						
						\item[] \begin{packed_item}
	
							\item[1.a] U trenutku odabira prijave, prijava više nije aktivna
							\item[] \begin{packed_enum}
								
								\item Sustav korisniku javlja da prijava više nije aktivna	
								\item Korisnik je preusmjeren na početnu stranicu
							\end{packed_enum}
						\end{packed_item}
					\end{packed_item}
					
					
				
				
					
				\subsubsection{Dijagrami obrazaca uporabe}
					
					\textit{Prikazati odnos aktora i obrazaca uporabe odgovarajućim UML dijagramom. Nije nužno nacrtati sve na jednom dijagramu. Modelirati po razinama apstrakcije i skupovima srodnih funkcionalnosti.}
				\eject		
				
			\subsection{Sekvencijski dijagrami}
				
				\textbf{\textit{dio 1. revizije}}\\
				
				
				\textit{Nacrtati sekvencijske dijagrame koji modeliraju najvažnije dijelove sustava (max. 4 dijagrama). Ukoliko postoji nedoumica oko odabira, razjasniti s asistentom. Uz svaki dijagram napisati detaljni opis dijagrama.}
				\\
				\\
				\textbf{TODO: ako treba prenijeti dokumentaciju u wiki pa nacrtati specificirane dijagrame obrazaca uporabe i sekvencijske}
				\\
				\\
				\textbf{Obrasci uporabe UC02, UC03, UC13, UC14 - unos nove prijave, pregled postojećih prijava, povezivanje na postojeću prijavu, uvid u aktivne prijave sa karte}\\
				Korisnik (prijavljeni ili anonimni) može podnijeti novu prijavu sustavu. Korisnik odabire opciju unosa nove prijave nakon čega poslužitelj vraća formu za upis podataka prijave. 
				Korisnik šalje prijavu te sustav provjerava unos korisnika i provjerava u postoje li vremenski i prostorno bliske aktivne prijave te ako postoje, vraća formu korisniku gdje se može spojiti na neku od bliskih prijava. 
				Korisnik šalje poslužitelju hoće li se i na koju povezati te poslužitelj sprema novu prijavu u bazu podataka. Korisnik također može poslati zahtjev za pregledom svih postojećih prijava uz opcionalne filtere po temi i lokaciji na što poslužitelj reagira dohvaćanjem tih podataka iz baze te slanjem istih korisniku. 
				Korisnik također može kliknuti na neku od aktivnih prijava na karti na što se poslužitelju šalje zahtjev na koji on reagira dohvaćanjem detalja te prijave iz baze podataka te odgovaranjem korisniku istima.
				
				
				\textbf{Obrazac uporabe UC01 - registracija korisnika u sustav}\\
				Neprijavljeni korisnik može poslati zahtjev za registracijom na što poslužitelj odgovara formom za registraciju. Kad korisnik pošalje podatke, poslužitelj ih validira te sprema u bazu podataka.

				\textbf{Obrazac uporabe UC05, UC06, UC07, UC10 - uređivanje podataka prijave, brisanje prijave, pregled korisnika, brisanje korisničkog računa, pregled gradskih ureda}\\
				Adminisistrator može poslati zahtjev poslužitelju za uređivanje podataka iz pojedine prijave na što poslužitelj zahtjeva podatke te prijave iz baze te ih u formi za uređivanje vraća administratoru. Nakon uređivanja administrator šalje nove podatke poslužitelju na što ih on sprema u bazu.
				Administrator također može poslati poslužitelju zahtjev za brisanjem prijave na što poslužitelj briše prijavu iz baze.
				Osim toga, administrator ima uvid i u sve registrirane korisnike kao liste koju na zahtjev poslužitelj zahtijeva iz baze te vraća administratoru.
				Osim korisnika, na isti način administrator može pregledati i gradske urede.

				\textbf{Obrazac uporabe UC04, UC08, UC11, UC15 - prijava korisnika u sustav, pregled osobnih podataka, pregled povijesti vlastitih prijava, mijenjanje sadržaja aktivne prijave}\\
				Registrirani korisnik može poslati zatjev za prijavom u sustav na što poslužitelj odgovara formom za prijavu. Kad korisnik prijavu pošalje, poslužitelj ju validira uz pomoć baze podataka te obaviještava korisnika o uspiješnosti prijave. Prijavljeni korisnik može poslati poslužitelju zahtjev za pregledom povijesti vlastitih prijava na što ih poslužitelj dohvaća iz baze podataka te ih vraća korisniku. Na isti način prijavljeni korisnik može pregledati osobne podatke.
				Ukoliko prijavljeni korisnik želi mijenati svoju aktivnu prijavu, može odabrati jednu iz liste aktivnih prijava na što se poslužitelju šalje zahtjev za uređivanjem iste. Poslužitelj tad dohvaća podatke te prijave iz baze podataka te korisniku šalje formu s podacima prijave. Korisnik tad uređuje podatke te ih šalje poslužitelju na što ih poslužitelj validira te sprema u bazu podataka.

				\eject
	
		\section{Ostali zahtjevi}
		
			\textbf{\textit{dio 1. revizije}}\\
		 
			 \textit{Nefunkcionalni zahtjevi i zahtjevi domene primjene dopunjuju funkcionalne zahtjeve. Oni opisuju \textbf{kako se sustav treba ponašati} i koja \textbf{ograničenja} treba poštivati (performanse, korisničko iskustvo, pouzdanost, standardi kvalitete, sigurnost...). Primjeri takvih zahtjeva u Vašem projektu mogu biti: podržani jezici korisničkog sučelja, vrijeme odziva, najveći mogući podržani broj korisnika, podržane web/mobilne platforme, razina zaštite (protokoli komunikacije, kriptiranje...)... Svaki takav zahtjev potrebno je navesti u jednoj ili dvije rečenice.}
			 
			 
			 
	