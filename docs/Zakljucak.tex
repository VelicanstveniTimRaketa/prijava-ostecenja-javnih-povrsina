\chapter{Zaključak i budući rad}
		
		Zadatak našeg tima Tim Raketa bio je proizvesti web aplikaciju kojom suvremeni građanin može uočenu javnu štetu skenirati i prijaviti. Izuzevši prva dva tjedna semestra, rad na aplikaciji i dokumentaciji redovno se vršio u nezanemarivoj satnici svakog tjedna. Projekt je bio podjeljen na dvije faze: \\
		\\
		\indent \textbf{1) Prva faza} je bila problematičnija jer se problem činio apastraktan, a oskudnog znanja za preporučene tehnologije se ne može reći da nije bilo. Svakako poznavanje članova već od prije je uvelike pridonijelo samoj podjeli rada na projektu i olakšanom dogovaranju oko izrade zadataka, ali i preporučivanja tehnologija za samu izvedbu. Tim se podijelio tako što su trojica preuzela ulogu developera na backend-u, dvojica na frontend-u te jedan na dokumentaciji projekta. Kako je prva faza stavila velik naglasak na dokumentaciju projekta tako se prema tome rad i odvijao. Pisanje dokumentacije je pridonijelo i samom razumijavanju problema jer je pomoću dijagrama obrazaca i sekvencijskih dijagrama služila kao smjernica kuda projekt treba ići. Prva je faza trajala do 17.11.2023. kada je i predana aplikacija s generičkim funkcionalnostima, no ono što je važnije jest da se na kraju ove faze cijeli tim približio sa tehnologijama i osjećao komfort u njihovom okruženju.\\
		\\
		\indent \textbf{2) Druga faza} krenula je odmah početkom prosinca, a naglasak je stavila nešto više ovaj put na funckionalnost aplikacije uz ponešto manje količine dokumentacije. Ovdje su članovi bili usmjereni da i najmanje sitnice dovedu do prihavtljivog oblika kako bi se aplikacija mogla pustiti u pogon.  Ovdje je kemija backend i frontend tima bila ključ napretka pošto je realizacija dijela jednog tima ovisila o implementaciji istog dijela kod drugog tima. Nakon realizacije funkcionalnosti bilo je potrebno dovršiti dokumentaciju traženim dijagramima i napraviti testove za važnije stvari da se ta funkcionalnost pokaže. Ova faza trajala je do 19.1.2024. kada je i finalna verzija uspješno objavljena.\\
		\\
		Na početku projekta se ovo sve činilo kao nemoguća misija zbog nepoznavanja tehnologija pa su sukladno tome, naizgled jednostavni zadaci prerasli u kompleksne. Na svu sreću u timu je bilo volje i ambicije za napredak pa je iz tjedna u tjedan anksioznost oko projekta nestajala i na kraju se odradio sav željeni dio sustava koji se planirao pa se može reći da je zadovoljstvo tima bilo prisutno. O stečenom iskustvu se može pričati samo u pozitivnom kontekstu. Naime, uistinu smo iz prve ruke okusili kako to sve izgleda, barem donekle, u stvarnom svijetu i radu u timskom okruženju. Naravno, nije sve bilo perfektno pa tako jedna veća mana bila nedostatak poznavanja kako načina pristupa problemu, tako i znanja u potrebnim tehnologijma. Ipak sve se to naposljetku prebrodilo i možemo reći da se sa stečenim znanjem i otvorenog uma može ići samo naprijed.
		\eject 