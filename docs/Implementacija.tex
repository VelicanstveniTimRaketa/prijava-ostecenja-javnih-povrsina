\chapter{Implementacija i korisničko sučelje}
		
		
		\section{Korištene tehnologije i alati}
		
			Dio  zadužen za frontend i komunikaciju s korisnikom je ostvaren pomoću knjižnice \href{https://react.dev/}{React}, te je pisan u programskom jeziku
			\href{https:www.typescriptlang.org/}{TypeScript}. React, kao bibilioteka je široko raspostranjen u području stvaranja korisničkog sučelja. Jedna od glavnih prednosti jest objedinjavanje visokih performansi brzine aplikacije, korištenjem 
virtualnog DOM-a, te omogućuje podršku za ponovnu uporabivost komponenti i nadogradnju sustava (zbog svoje enkapsuliranosti). Razvila ga je grupa Meta (Facebook) i naglask se stavlja na njegovu pripadnost knjižnicama, a ne radnim okvirima. \\
			Razvojno okruženje u kojem je pisan cijeli frotend je \href{https://code.visualstudio.com/}{Visual Studio Code}(tzv. VSC). VSC je integrirano razvojno okuženje u vlasništvu tvrtke Microsoft. Danas je to jedna od najkorišenijih tehnologija za uređivanje i pisanje koda, a to je s dobrim razlogom jer on nudi mnoštva ekstenzija za razne programske jezike (pa tako i za TypeScript u kojem je naša aplikacija pisana), ali i jednostavne funkcionalnosti koje ispadaju uvelike korisne te olakšavaju posao pisanja koda, a neke od tih su: debugger, IntelliSense koji služi kao autofill i ugrađena podrška za Git. \\
			Backend sustava pisan je u programskom jeziku \href{https://www.java.com/en/}{Java} korištenjem radnog okvira \href{https://spring.io/projects/spring-boot/}{Spring Boot}. Spring Boot je implementiran kao specijalizacija generaliziranijeg radnog okvira Spring, a omogućuje izradu stand-alone aplikacija. Dobrim dijelom olakšava posao izgradnje sustava jer sam konfigurira funckionalnosti Springa, ali i sam ima već podešene one dijelove web aplikacije koji se učestalo koriste (npr. servleti).\\
			Cijeli backend je pisan u razvojnom okruženju \href{https://www.jetbrains.com/idea/}{InntelliJ IDEA}. To je integrirano razvojno sučelje u vlasništvu kompanije JetBrains, a prevladavajuće je u području pisanja porgrama u Javi i Kotlinu. Velike funkcionalnosti koje koristi su coding assistance, remote cooperation kao i jednostavnost uporabe. \\
			Za dokumentaciju se koristio programski jezik \href{https://www.latex-project.org//}{$LaTeX$}. $LaTeX$ je jezik za pisanje strukturiranih tesktova, piše se kao običan tekst sa dodanom semantičkom strukturom (ne prikazuje se kao konačan proizvod kao npr. Microsoft Word) što mu omogućuje stabilniji rad, a sve što zahtjeva je instalriana distribucija TeX-a; što je u našem slučaju \href{https://miktex.org/}{MiKTeX}. Dokumentacija je pisana u uređivaču teksta zvanom \href{https://www.xm1math.net/texmaker/}{Texmaker} koji se ističe po jednostavnosti pisanju $LaTeX$ dokumenata kao i u svojoj dostupnosti.\\
			U izradi UML dijagrama korištene su dvije tehnologije: \href{https://astah.net/}{Astah} i \href{https://www.visual-paradigm.com/}{Visual Paradigm}. Oba se alata ističu po svojoj rasprostranjenosti tako što omogućavaju izbor kreiranja svakojakih dijagrama, ali i po svojoj jednostavnosti. Razlika je u tome što je Astah porgram koje se pokreće lokalno na računalu, a Visual Paradigm se pokreće online i sprema trenutne izmjene po izboru lokalno ili na Cloud.\\
			Kako bi se lakše upravljalo verzijama projekta, korišten je \href{https://git-scm.com/}{Git}. Git je besplatan, open source sustav koji se koristi za upravljanje kako manjih, tako i većih projekata. Prednosti su mu brzina, jednostavnost i lakoća upravljanja projektima u timskom radu. Vanjski repozitorij projekta se nalazi na besplatnoj web platformi \href{https://github.com/}{GitHub} koja omogućuje lako upravljanje projektom svim sudionicima repozitorija. \\
			Za produkciju aplikacije korišen je cloud sustav \href{https://render.com/}{Render}. Render olakšava puštanje aplikacija u pogon kako bi bile javno dostupne za pronalazak an internetu. \\
			Kako bi se u timu maksimalno olakšala komunikacija članova kroišen je \href{https://discord.com/}{Discord}. Discord je društvena platforma u kojoj je naglasak na jednostavnosti postizanja glasovne, video i tekstne komunikacije u zajednicama koje se nazivaju serveri kojima se pristupa pomoću poveznice, a omogućuju kavlitetan integritet tima (zajednice).
			\eject 
		
	
		\section{Ispitivanje programskog rješenja}
			
			\textbf{\textit{dio 2. revizije}}\\
			
			 \textit{U ovom poglavlju je potrebno opisati provedbu ispitivanja implementiranih funkcionalnosti na razini komponenti i na razini cijelog sustava s prikazom odabranih ispitnih slučajeva. Studenti trebaju ispitati temeljnu funkcionalnost i rubne uvjete.}
	
			
			\subsection{Ispitivanje komponenti}
			\textit{Potrebno je provesti ispitivanje jedinica (engl. unit testing) nad razredima koji implementiraju temeljne funkcionalnosti. Razraditi \textbf{minimalno 6 ispitnih slučajeva} u kojima će se ispitati redovni slučajevi, rubni uvjeti te izazivanje pogreške (engl. exception throwing). Poželjno je stvoriti i ispitni slučaj koji koristi funkcionalnosti koje nisu implementirane. Potrebno je priložiti izvorni kôd svih ispitnih slučajeva te prikaz rezultata izvođenja ispita u razvojnom okruženju (prolaz/pad ispita). }
			
			
			
			\subsection{Ispitivanje sustava}
			
			 \textit{Potrebno je provesti i opisati ispitivanje sustava koristeći radni okvir Selenium\footnote{\url{https://www.seleniumhq.org/}}. Razraditi \textbf{minimalno 4 ispitna slučaja} u kojima će se ispitati redovni slučajevi, rubni uvjeti te poziv funkcionalnosti koja nije implementirana/izaziva pogrešku kako bi se vidjelo na koji način sustav reagira kada nešto nije u potpunosti ostvareno. Ispitni slučaj se treba sastojati od ulaza (npr. korisničko ime i lozinka), očekivanog izlaza ili rezultata, koraka ispitivanja i dobivenog izlaza ili rezultata.\\ }
			 
			 \textit{Izradu ispitnih slučajeva pomoću radnog okvira Selenium moguće je provesti pomoću jednog od sljedeća dva alata:}
			 \begin{itemize}
			 	\item \textit{dodatak za preglednik \textbf{Selenium IDE} - snimanje korisnikovih akcija radi automatskog ponavljanja ispita	}
			 	\item \textit{\textbf{Selenium WebDriver} - podrška za pisanje ispita u jezicima Java, C\#, PHP koristeći posebno programsko sučelje.}
			 \end{itemize}
		 	\textit{Detalji o korištenju alata Selenium bit će prikazani na posebnom predavanju tijekom semestra.}
			
			\eject 
		
		
		\section{Dijagram razmještaja}
			
			\textbf{\textit{dio 2. revizije}}
			
			 \textit{Potrebno je umetnuti \textbf{specifikacijski} dijagram razmještaja i opisati ga. Moguće je umjesto specifikacijskog dijagrama razmještaja umetnuti dijagram razmještaja instanci, pod uvjetom da taj dijagram bolje opisuje neki važniji dio sustava.}
			
			\eject 
		
		\section{Upute za puštanje u pogon}
		
			\textbf{\textit{dio 2. revizije}}\\
		
			 \textit{U ovom poglavlju potrebno je dati upute za puštanje u pogon (engl. deployment) ostvarene aplikacije. Na primjer, za web aplikacije, opisati postupak kojim se od izvornog kôda dolazi do potpuno postavljene baze podataka i poslužitelja koji odgovara na upite korisnika. Za mobilnu aplikaciju, postupak kojim se aplikacija izgradi, te postavi na neku od trgovina. Za stolnu (engl. desktop) aplikaciju, postupak kojim se aplikacija instalira na računalo. Ukoliko mobilne i stolne aplikacije komuniciraju s poslužiteljem i/ili bazom podataka, opisati i postupak njihovog postavljanja. Pri izradi uputa preporučuje se \textbf{naglasiti korake instalacije uporabom natuknica} te koristiti što je više moguće \textbf{slike ekrana} (engl. screenshots) kako bi upute bile jasne i jednostavne za slijediti.}
			
			
			 \textit{Dovršenu aplikaciju potrebno je pokrenuti na javno dostupnom poslužitelju. Studentima se preporuča korištenje neke od sljedećih besplatnih usluga: \href{https://aws.amazon.com/}{Amazon AWS}, \href{https://azure.microsoft.com/en-us/}{Microsoft Azure} ili \href{https://www.heroku.com/}{Heroku}. Mobilne aplikacije trebaju biti objavljene na F-Droid, Google Play ili Amazon App trgovini.}
			
			
			\eject 